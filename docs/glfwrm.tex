%-------------------------------------------------------------------------
% GLFW Reference Manual
% API Version: 2.4
% $Id: glfwrm.tex,v 1.1 2003-11-03 21:23:21 marcus256 Exp $
%-------------------------------------------------------------------------

% Document class
\documentclass[a4paper,11pt,oneside]{report}

% Document title and API version
\newcommand{\glfwdoctype}[1][0]{Reference Manual}
\newcommand{\glfwapiver}[1][0]{2.4}

% Common document settings and macros
\input{glfwdoc.sty}

% PDF specific document settings
\hypersetup{pdftitle={GLFW Reference Manual}}
\hypersetup{pdfauthor={Marcus Geelnard}}
\hypersetup{pdfkeywords={GLFW,OpenGL,reference}}


%-------------------------------------------------------------------------
% Document body
%-------------------------------------------------------------------------

\begin{document}

% Title page
\glfwmaketitle

% Summary, trademarks and table of contents
\pagenumbering{roman}
\setcounter{page}{1}

%-------------------------------------------------------------------------
% Summary and Trademarks
%-------------------------------------------------------------------------
\chapter*{Summary}

This document is a function reference manual for the \GLFW\ API. For a
description of how to use \GLFW\ you should refer to the \textit{GLFW Users Guide}.
\vspace{10cm}

\large
Trademarks

\small
OpenGL and IRIX are registered trademarks of Silicon Graphics, Inc.\linebreak
Microsoft, Windows and MS�-DOS are registered trademarks of Microsoft Corporation.\linebreak
Mac OS is a registered trademark of Apple Computer, Inc.\linebreak
Linux is a registered trademark of Linus Torvalds.\linebreak
FreeBSD is a registered trademark of Wind River Systems, Inc.\linebreak
Solaris is a trademark of Sun Microsystems, Inc.\linebreak
UNIX is a registered trademark of The Open Group.\linebreak
X Window System is a trademark of The Open Group.\linebreak
POSIX is a trademark of IEEE.\linebreak
Truevision, TARGA and TGA are registered trademarks of Truevision, Inc.\linebreak

All other trademarks mentioned in this document are the property of their respective owners.
\normalsize


%-------------------------------------------------------------------------
% Table of contents
%-------------------------------------------------------------------------
\tableofcontents
\pagebreak

% Document chapters starts here...
\pagenumbering{arabic}
\setcounter{page}{1}


%-------------------------------------------------------------------------
% Introduction
%-------------------------------------------------------------------------
\chapter{Introduction}
\thispagestyle{fancy}

\GLFW\ is a portable API (Application Program Interface) that handles
operating system specific tasks related to \OpenGL\ programming. While
\OpenGL\ in general is portable, easy to use and often results in tidy and
compact code, the operating system specific mechanisms that are required
to set up and manage an \OpenGL\ window are quite the opposite. \GLFW\ tries
to remedy this by providing the following functionality:

\begin{itemize}
\item Opening and managing an \OpenGL\ window.
\item Keyboard, mouse and joystick input.
\item A high precision timer.
\item Multi threading support.
\item Support for querying and using \OpenGL\ extensions.
\item Image file loading support.
\end{itemize}
\vspace{18pt}

All this functionality is implemented as a set of easy-to-use functions,
which makes it possible to write an \OpenGL\ application framework in just a
few lines of code. The \GLFW\ API is completely operating system and
platform independent, which makes it very simple to port \GLFW\ based \OpenGL\
applications to a variety of platforms.

Currently supported platforms are:
\begin{itemize}
\item Microsoft Windows\textsuperscript{\textregistered} 95/98/ME/NT/2000/XP/.NET Server.
\item Unix\textsuperscript{\textregistered} or Unix�-like systems running the
X Window System\texttrademark, e.g. Linux\textsuperscript{\textregistered},
IRIX\textsuperscript{\textregistered}, FreeBSD\textsuperscript{\textregistered},
Solaris\texttrademark, QNX\textsuperscript{\textregistered} and
Mac OS\textsuperscript{\textregistered} X.
\item Mac OS\textsuperscript{\textregistered} X (Carbon)\footnote{Only a subset of the \GLFW\ API is supported for this platform at the time of writing.}
\item AmigaOS\footnotemark[\value{footnote}]
\end{itemize}



%-------------------------------------------------------------------------
% Function Reference
%-------------------------------------------------------------------------
\chapter{Function Reference}
\thispagestyle{fancy}

%-------------------------------------------------------------------------
\section{GLFW Initialization and Termination}
Before any \GLFW\ functions can be used, \GLFW\ must be initialized to
ensure proper functionality, and before a program terminates, \GLFW\ has to
be terminated in order to free up resources etc.


%-------------------------------------------------------------------------
\subsection{glfwInit}

\textbf{C language syntax}
\begin{lstlisting}
int glfwInit( void )
\end{lstlisting}

\textbf{Parameters}
\\none

\begin{refreturn}
If the function succeeds, GL\_TRUE is returned.\\
If the function fails, GL\_FALSE is returned.
\end{refreturn}

\begin{refdescription}
The glfwInit function initializes \GLFW. No other \GLFW\ functions may be
used before this function has been called.
\end{refdescription}

\begin{refnotes}
This function may take several seconds to complete on some systems, while
on other systems it may take only a fraction of a second to complete.
\end{refnotes}


%-------------------------------------------------------------------------
\subsection{glfwTerminate}

\textbf{C language syntax}
\begin{lstlisting}
void glfwTerminate( void )
\end{lstlisting}

\textbf{Parameters}
\\none

\begin{refreturn}
none
\end{refreturn}

\begin{refdescription}
The function terminates \GLFW. Among other things it closes the window,
if it is opened, and kills any running threads. This function must be
called before a program exits.
\end{refdescription}


%-------------------------------------------------------------------------
\subsection{glfwGetVersion}

\textbf{C language syntax}
\begin{lstlisting}
void glfwGetVersion( int *major, int *minor, int *rev )
\end{lstlisting}

\textbf{Parameters}
\begin{description}
\item [\textit{major}]\ \\
  Pointer to an integer that will hold the major version number.
\item [\textit{minor}]\ \\
  Pointer to an integer that will hold the minor version number.
\item [\textit{rev}]\ \\
  Pointer to an integer that will hold the revision.
\end{description}

\begin{refreturn}
The function returns the major and minor version numbers and the revision
for the currently linked \GLFW\ library.
\end{refreturn}

\begin{refdescription}
The function returns the \GLFW\ library version.
\end{refdescription}


%-------------------------------------------------------------------------
\pagebreak
\section{Window Handling}
The main functionality of \GLFW\ is to provide a simple interface to
\OpenGL\ window management. \GLFW\ can open one window, which can be
either a normal desktop window or a fullscreen window.


%-------------------------------------------------------------------------
\subsection{glfwOpenWindow}

\textbf{C language syntax}
\begin{lstlisting}
int glfwOpenWindow( int width, int height, int redbits,
    int greenbits, int bluebits, int alphabits, int depthbits,
    int stencilbits, int mode )
\end{lstlisting}

\textbf{Parameters}
\begin{description}
\item [\textit{width}]\ \\
  The width of the window. If \textit{width} is zero, it will be calculated
  as \textit{4/3*height} if \textit{height} is not zero. If both \textit{width}
  and \textit{height} are zero, \textit{width} will be set to 640.
\item [\textit{hieght}]\ \\
  The height of the window. If \textit{height} is zero, it will be calculated as
  \textit{3/4*width} if \textit{width} is not zero. If both \textit{width} and
  \textit{height} are zero, \textit{height} will be set to 480.
\item [\textit{redbits, greenbits, bluebits}]\ \\
  The number of bits to use for each color component of the color buffer (0 means
  default color depth). For instance, setting \textit{redbits=5, greenbits=6,
  and bluebits=5} will generate a 16-�bit color buffer, if possible.
\item [\textit{alphabits}]\ \\
  The number of bits to use for the alpha buffer (0 means no alpha buffer).
\item [\textit{depthbits}]\ \\
  The number of bits to use for the depth buffer (0 means no depth buffer).
\item [\textit{stencilbits}]\ \\
  The number of bits to use for the stencil buffer (0 means no stencil buffer).
\item [\textit{mode}]\ \\
  Selects which type of \OpenGL\ window to use. \textit{mode} can be either
  GLFW\_WINDOW, which will generate a normal desktop window, or GLFW\_FULLSCREEN,
  which will generate a window which covers the entire screen. When
  GLFW\_FULLSCREEN is selected, the video mode will be changed to the resolution
  that closest matches the \textit{width} and \textit{height} parameters.
\end{description}

\begin{refreturn}
If the function succeeds, GL\_TRUE is returned.\\
If the function fails, GL\_FALSE is returned.
\end{refreturn}

\begin{refdescription}
The function opens a window that best matches the parameters given to the function.
How well the resulting window matches the desired window depends mostly on the
available hardware and \OpenGL\ drivers. In general, selecting a fullscreen mode has
better chances of generating a close match than does a normal desktop window, since
\GLFW\ can freely select from all the available video modes. A desktop window is
normally restricted to the video mode of the desktop.
\end{refdescription}

\begin{refnotes}
For additional control of window properties, see \textbf{glfwOpenWindowHint}.

In fullscreen mode the mouse cursor is hidden by default, and any system screensavers
are prohibited from starting. In windowed mode the mouse cursor is visible, and
screensavers are allowed to start. To change the visibility of the mouse cursor, use
\textbf{glfwEnable} or \textbf{glfwDisable} with the argument GLFW\_MOUSE\_CURSOR.

In order to determine the actual properties of an opened window, use
\textbf{glfwGetWindowParam} and \textbf{glfwGetWindowSize} (or
\textbf{glfwSetWindowSizeCallback}).
\end{refnotes}


%-------------------------------------------------------------------------
\subsection{glfwOpenWindowHint}

\textbf{C language syntax}
\begin{lstlisting}
void glfwOpenWindowHint( int target, int hint )
\end{lstlisting}

\textbf{Parameters}
\begin{description}
\item [\textit{target}]\ \\
  Can be any of the constants in the table below:

\begin{tabular}{|l|p{8cm}|} \hline \raggedright
\textbf{Name}            & \textbf{Description} \\ \hline
GLFW\_REFRESH\_RATE      & Vertical monitor refresh rate in Hz (only used for fullscreen windows).\\ \hline
GLFW\_ACCUM\_RED\_BITS   & Number of bits for the red channel of the accumulator buffer.\\ \hline
GLFW\_ACCUM\_GREEN\_BITS & Number of bits for the green channel of the accumulator buffer.\\ \hline
GLFW\_ACCUM\_BLUE\_BITS  & Number of bits for the blue channel of the accumulator buffer.\\ \hline
GLFW\_ACCUM\_ALPHA\_BITS & Number of bits for the alpha channel of the accumulator buffer.\\ \hline
GLFW\_AUX\_BUFFERS       & Number of auxiliary buffers.\\ \hline
GLFW\_STEREO             & Specify if stereo rendering should be supported (can be GL\_TRUE or GL\_FALSE).\\ \hline
\end{tabular}

\item [\textit{hint}]\ \\
  An integer giving the value of the corresponding target (see table above).
\end{description}

\begin{refreturn}
none
\end{refreturn}

\begin{refdescription}
The function sets additional properties for a window that is to be opened.
For a hint to be registered, the function must be called before calling
\textbf{glfwOpenWindow}. When the \textbf{glfwOpenWindow} function is
called, any hints that were registered with the \textbf{glfwOpenWindowHint}
function are used for setting the corresponding window properties, and
then all hints are reset to "do not care".
\end{refdescription}

\begin{refnotes}
In order to determine the actual properties of an opened window, use
\textbf{glfwGetWindowParam} (after the window has been opened).

GLFW\_STEREO is a hard constraint. If stereo rendering is requested, but
no stereo rendering capable pixel formats / visuals are available,
\textbf{glfwOpenWindow} will fail.

GLFW\_REFRESH\_RATE is only supported under Windows.

The GLFW\_REFRESH\_RATE property should be used with caution. Most
systems have default values for monitor refresh rates that are optimal
for the specific system. Specifying the refresh rate can override these
settings, which can result in suboptimal operation. The monitor may be
unable to display the resulting video signal, or in the worst case it may
even be damaged!

\end{refnotes}


%-------------------------------------------------------------------------
\subsection{glfwCloseWindow}

\textbf{C language syntax}
\begin{lstlisting}
void glfwCloseWindow( void )
\end{lstlisting}

\textbf{Parameters}
\\none

\begin{refreturn}
none
\end{refreturn}

\begin{refdescription}
The function closes an opened window and destroys the associated \OpenGL\
context.
\end{refdescription}


%-------------------------------------------------------------------------
\subsection{glfwSetWindowTitle}

\textbf{C language syntax}
\begin{lstlisting}
void glfwSetWindowTitle( const char *title )
\end{lstlisting}

\textbf{Parameters}
\begin{description}
\item [\textit{title}]\ \\
  Pointer to a  null terminated ISO 8859-1 (8-bit Latin 1) string that
  holds the title of the window.
\end{description}

\begin{refreturn}
none
\end{refreturn}

\begin{refdescription}
The function changes the title of the opened window.
\end{refdescription}

\begin{refnotes}
The title property of a window is often used in situations other than for
the window title, such as the title of an application icon when it is in
iconified state.
\end{refnotes}


%-------------------------------------------------------------------------
\subsection{glfwSetWindowSize}

\textbf{C language syntax}
\begin{lstlisting}
void glfwSetWindowSize( int width, int height )
\end{lstlisting}

\textbf{Parameters}
\begin{description}
\item [\textit{width}]\ \\
  Width of the window.
\item [\textit{height}]\ \\
  Height of the window.
\end{description}

\begin{refreturn}
none
\end{refreturn}

\begin{refdescription}
The function changes the size of an opened window, and calls the window
size callback function (if one is registered). If the window is in
fullscreen mode, the video mode will be changed to a resolution that
closest matches the width and height parameters (the number of color
bits will not be changed).
\end{refdescription}

\begin{refnotes}
The \OpenGL\ context is guaranteed to be preserved after calling
\textbf{glfwSetWindowSize}, even if the video mode is changed.

Changing the size of a fullscreen window is not supported under AmigaOS,
since that would destroy the associated \OpenGL\ context.
\end{refnotes}


%-------------------------------------------------------------------------
\subsection{glfwSetWindowPos}

\textbf{C language syntax}
\begin{lstlisting}
void glfwSetWindowPos( int x, int y )
\end{lstlisting}

\textbf{Parameters}
\begin{description}
\item [\textit{x}]\ \\
  Horizontal position of the window, relative to the upper left corner
  of the desktop.
\item [\textit{y}]\ \\
  Vertical position of the window, relative to the upper left corner of
  the desktop.
\end{description}

\begin{refreturn}
none
\end{refreturn}

\begin{refdescription}
The function changes the position of an opened window. It does not have
any effect on a fullscreen window.
\end{refdescription}


%-------------------------------------------------------------------------
\subsection{glfwGetWindowSize}

\textbf{C language syntax}
\begin{lstlisting}
void glfwGetWindowSize( int *width, int *height )
\end{lstlisting}

\textbf{Parameters}
\begin{description}
\item [\textit{width}]\ \\
  Pointer to an integer that will hold the width of the window.
\item [\textit{height}]\ \\
  Pointer to an integer that will hold the height of the window.
\end{description}

\begin{refreturn}
The current width and height of the opened window is returned in the
\textit{width} and \textit{height} parameters, respectively.
\end{refreturn}

\begin{refdescription}
The function is used for determining the size of an opened window.
The returned values are dimensions of the client area of the window
(i.e. excluding any window borders and decorations).
\end{refdescription}

\begin{refnotes}
Even if the size of a fullscreen window does not change once the window
has been opened, it does not necessarily have to be the same as the size
that was requested using \textbf{glfwOpenWindow}. Therefor it is wise to
use this function to determine the true size of the window once it has
been opened.
\end{refnotes}


%-------------------------------------------------------------------------
\subsection{glfwSetWindowSizeCallback}

\textbf{C language syntax}
\begin{lstlisting}
void glfwSetWindowSizeCallback( GLFWwindowsizefun cbfun )
\end{lstlisting}

\textbf{Parameters}
\begin{description}
\item [\textit{cbfun}]\ \\
  Pointer to a callback function that will be called every time the
  window size changes. The function should have the following C language
  prototype:

  \textbf{void GLFWCALL functionname( int width, int height );}

  Where \textit{functionname} is the name of the callback function, and
  \textit{width} and \textit{height} are the window dimension passed to
  the function.

  If \textit{cbfun} is NULL, any previously selected callback function
  will be deselected.
\end{description}

\begin{refreturn}
none
\end{refreturn}

\begin{refdescription}
The function selects which function to be called upon a window size
change event.
\end{refdescription}

\begin{refnotes}
Window size changes are recorded continuously, but only reported when
\textbf{glfwPollEvents} or \textbf{glfwSwapBuffers} is called.
\end{refnotes}


%-------------------------------------------------------------------------
\subsection{glfwGetWindowParam}

\textbf{C language syntax}
\begin{lstlisting}
int glfwGetWindowParam( int param )
\end{lstlisting}

\textbf{Parameters}
\begin{description}
\item [\textit{param}]\ \\
  A token selecting which parameter the function should return (see below).
\end{description}

\begin{refreturn}
The function returns different parameters depending on the value of
\textit{param}. Below is a table of valid \textit{param} values, and
their corresponding return values:

\begin{tabular}{|l|p{9cm}|} \hline \raggedright
\textbf{Name}            & \textbf{Description} \\ \hline
GLFW\_OPENED             & GL\_TRUE if window is opened, else GL\_FALSE.\\ \hline
GLFW\_ACTIVE             & GL\_TRUE if window has focus, else GL\_FALSE.\\ \hline
GLFW\_ICONIFIED          & GL\_TRUE if window is iconified/minimized, else GL\_FALSE.\\ \hline
GLFW\_ACCELERATED        & GL\_TRUE if window is hardware accelerated.\\ \hline
GLFW\_RED\_BITS          & Number of bits for the red color component.\\ \hline
GLFW\_GREEN\_BITS        & Number of bits for the green color component.\\ \hline
GLFW\_BLUE\_BITS         & Number of bits for the blue color component.\\ \hline
GLFW\_ALPHA\_BITS        & Number of bits for the alpha buffer.\\ \hline
GLFW\_DEPTH\_BITS        & Number of bits for the depth buffer.\\ \hline
GLFW\_STENCIL\_BITS      & Number of bits for the stencil buffer.\\ \hline
GLFW\_REFRESH\_RATE      & Vertical monitor refresh rate in Hz. Zero indicates an unknown or a default refresh rate.\\ \hline
GLFW\_ACCUM\_RED\_BITS   & Number of bits for the red channel of the accumulator buffer.\\ \hline
GLFW\_ACCUM\_GREEN\_BITS & Number of bits for the green channel of the accumulator buffer.\\ \hline
GLFW\_ACCUM\_BLUE\_BITS  & Number of bits for the blue channel of the accumulator buffer.\\ \hline
GLFW\_ACCUM\_ALPHA\_BITS & Number of bits for the alpha channel of the accumulator buffer.\\ \hline
GLFW\_AUX\_BUFFERS       & Number of auxiliary buffers.\\ \hline
GLFW\_STEREO             & GL\_TRUE if stereo rendering is supported, else GL\_FALSE.\\ \hline
\end{tabular}
\end{refreturn}

\begin{refdescription}
The function is used for acquiring various properties of an opened window.
\end{refdescription}

\begin{refnotes}
GLFW\_ACCELERATED is only supported under Windows. Other systems will always
return GL\_TRUE. Under Windows, GLFW\_ACCELERATED means that the \OpenGL\
renderer is a 3rd party renderer, rather than the fallback Microsoft software
\OpenGL\ renderer. In other words, it is not a real guarantee that the \OpenGL\
renderer is actually hardware accelerated.

GLFW\_REFRESH\_RATE is only supported under Windows, XFree86 and AmigaOS.
Other systems will always return zero (0). With some Windows drivers, zero
(0) may be returned, indicating a default refresh rate.
\end{refnotes}


%-------------------------------------------------------------------------
\subsection{glfwSwapBuffers}

\textbf{C language syntax}
\begin{lstlisting}
void glfwSwapBuffers( void )
\end{lstlisting}

\textbf{Parameters}
\\ none

\begin{refreturn}
none
\end{refreturn}

\begin{refdescription}
The function swaps the back and front color buffers of the window. If
GLFW\_AUTO\_POLL\_EVENTS is enabled (which is the default),
\textbf{glfwPollEvents} is called before swapping the front and back
buffers.
\end{refdescription}


%-------------------------------------------------------------------------
\subsection{glfwSwapInterval}

\textbf{C language syntax}
\begin{lstlisting}
void glfwSwapInterval( int interval )
\end{lstlisting}

\textbf{Parameters}
\begin{description}
\item [\textit{interval}]\ \\
  Minimum number of monitor vertical retraces between each buffer swap
  performed by \textbf{glfwSwapBuffers}. If \textit{interval} is zero,
  buffer swaps will not be synchronized to the vertical refresh of the
  monitor (also known as 'VSync off').
\end{description}

\begin{refreturn}
none
\end{refreturn}

\begin{refdescription}
The function selects the minimum number of monitor vertical retraces that
should occur between two buffer swaps. If the selected swap interval is
one, the rate of buffer swaps will never be higher than the vertical
refresh rate of the monitor. If the selected swap interval is zero, the
rate of buffer swaps is only limited by the speed of the software and
the hardware.
\end{refdescription}

\begin{refnotes}
This function will only have an effect on hardware and drivers that
support user selection of the swap interval.
\end{refnotes}


%-------------------------------------------------------------------------
\pagebreak
\section{Video Modes}
Since \GLFW\ supports video mode changes when using a fullscreen window,
it also provides functionality for querying which video modes are
supported on a system.


%-------------------------------------------------------------------------
\subsection{glfwGetVideoModes}

\textbf{C language syntax}
\begin{lstlisting}
int glfwGetVideoModes( GLFWvidmode *list, int maxcount )
\end{lstlisting}

\textbf{Parameters}
\begin{description}
\item [\textit{list}]\ \\
  A vector of \textit{GLFWvidmode} structures, which will be filled out
  by the function.
\item [\textit{maxcount}]\ \\
  Maximum number of video modes that \textit{list} vector can hold.
\end{description}

\begin{refreturn}
The function returns the number of detected video modes (this number
will never exceed \textit{maxcount}). The \textit{list} vector is
filled out with the video modes that are supported by the system.
\end{refreturn}

\begin{refdescription}
The function returns a list of supported video modes. Each video mode is
represented by a \textit{GLFWvidmode} structure, which has the following
definition:

\begin{lstlisting}
typedef struct {
    int Width, Height; // Video resolution
    int RedBits;       // Number of red bits
    int GreenBits;     // Number of green bits
    int BlueBits;      // Number of blue bits
} GLFWvidmode;
\end{lstlisting}
\end{refdescription}

\begin{refnotes}
The returned list is sorted, first by color depth (\textit{RedBits} +
\textit{GreenBits} + \textit{BlueBits}), and then by resolution
(\textit{Width}*\textit{Height}), with the lowest resolution, fewest
bits per pixel mode first.
\end{refnotes}


%-------------------------------------------------------------------------
\subsection{glfwGetDesktopMode}

\textbf{C language syntax}
\begin{lstlisting}
void glfwGetDesktopMode( GLFWvidmode *mode )
\end{lstlisting}

\textbf{Parameters}
\begin{description}
\item [\textit{mode}]\ \\
  Pointer to a \textit{GLFWvidmode} structure, which will be filled out
  by the function.
\end{description}

\begin{refreturn}
The \textit{GLFWvidmode} structure pointed to by \textit{mode} is filled
out with the desktop video mode.
\end{refreturn}

\begin{refdescription}
The function returns the desktop video mode in a \textit{GLFWvidmode}
structure. See \textbf{glfwGetVideoModes} for a definition of the
\textit{GLFWvidmode} structure.
\end{refdescription}

\begin{refnotes}
The color depth of the desktop display is always reported as the number
of bits for each individual color component (red, green and blue), even
if the desktop is not using an RGB or RGBA color format. For instance, an
indexed 256 color display may report \textit{RedBits} = 3,
\textit{GreenBits} = 3 and \textit{BlueBits} = 2, which adds up to 8 bits
in total.

The desktop video mode is the video mode used by the desktop, \textit{not}
the current video mode (which may differ from the desktop video mode if
the \GLFW\ window is a fullscreen window).
\end{refnotes}


%-------------------------------------------------------------------------
\pagebreak
\section{Input Handling}
\GLFW\ supports three channels of user input: keyboard input, mouse input
and joystick input.

Keyboard and mouse input can be treated either as events, using callback
functions, or as state, using functions for polling specific keyboard and
mouse states. Regardless of which method is used, all keyboard and mouse
input is collected using window event polling.

Joystick input is asynchronous to the keyboard and mouse input, and does
not require event polling for keeping up to date joystick information.
Also, joystick input is independent of any window, so a window does not
have to be opened for joystick input to be used.


%-------------------------------------------------------------------------
\subsection{glfwPollEvents}

\textbf{C language syntax}
\begin{lstlisting}
void glfwPollEvents( void )
\end{lstlisting}

\textbf{Parameters}
\\ none

\begin{refreturn}
none
\end{refreturn}

\begin{refdescription}
The function is used for polling for events, such as user input and
window resize events. Upon calling this function, all window state and
keyboard and mouse input state is updated. If any related callback
functions are registered, these are called during the call to
\textbf{glfwPollEvents}.
\end{refdescription}

\begin{refnotes}
\textbf{glfwPollEvents} is called implicitly from \textbf{glfwSwapBuffers}
if GLFW\_AUTO\_POLL\_EVENTS is enabled (default). Thus, if
\textbf{glfwSwapBuffers} is called frequently, which is normally the case,
there is no need to call \textbf{glfwPollEvents}.
\end{refnotes}


%-------------------------------------------------------------------------
\subsection{glfwX}

\textbf{C language syntax}
\begin{lstlisting}
void glfwX( void )
\end{lstlisting}

\textbf{Parameters}
\begin{description}
\item [\textit{par1}]\ \\
  Text1.
\end{description}

\begin{refreturn}
none
\end{refreturn}

\begin{refdescription}
x
\end{refdescription}

\begin{refnotes}
none
\end{refnotes}


\end{document}
